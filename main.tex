\documentclass{article}
\usepackage[utf8]{inputenc}

\title{Critical Computer Science Reading List}
\author{\textbf{Audrey Beard} \\
        beardj2@rpi.edu \\
        Computer Science Department \\
        Rensselaer Polytechnic Institute}
\date{\today}

\usepackage{natbib}
\usepackage{graphicx}

\begin{document}

\maketitle

\section{Introduction}
    The intent of this document is to provide context for Critical Computer Science Pedagogy at Rensselaer Polytechnic Institute.
    Being rooted in computer vision and machine learning, I undoubtedly bias my literature review towards those topics.
    This document is broken into several sections, ranging from "pure" pedagogical theory, to science and technology studies (STS) literature, to "pure" computer science papers.
    
\section{Critiques of CS and Related}
    Ben Green's paper \textit{``Good'' Isn't Good Enough}\cite{greenGoodIsnGood2019} is an incisive critique of data science practice, and represents his views that data scientists must think and act critically when performing and thinking about data science.
    He insists that we cannot simply rely on ``good'' or ``ethical'' to guide their work. ``Social good'' in particular is a slippery topic that does not have rigorous definitions and so is often used flexibly to suit business needs.
    This is part of a larger working manifesto of sorts called \textit{Data Science as Political Action}\cite{greenDataSciencePolitical2019}.
    This paper won the ``Best Paper Award'' in the Public Policy Track at the 2019 NeurIPS Joint Workshop on AI for Social Good.

\section{Critical Computer Science Research}
    Skirpan and Yeh's 2017 paper \textit{Designing a Moral Compass for the Future of Computer Vision Using Speculative Analysis }\cite{skirpanDesigningMoralCompass2017} is concerned with grounding ethics for computer vision in an analysis of potential consequences of a given technology. Their use of \texit{speculative analysis}, a method for exploring a scenario that has not yet occurred via realistic fiction, is particularly notable.
    This tactic is rarely seen in traditional computer vision research, though it is at home in the CVPR Workshops, having appeared in \textit{The Bright and Dark Sides of Computer Vision: Challenges and Opportunities for Privacy and Security}.
    
\section{Critical Engagement from Outside CS}
    Langdon Winner's seminal essay \textit{Do Algorithms Have Politics?}\cite{winnerArtifactsHavePolitics1980} explores the claim implied by the title. He posits that artifacts have politics in two distinct ways: as being designed explicitly for some purpose, which reflects the politics of the inventor (the atom bomb, corsets, the Internet, etc.), and as being implicitly reliant \textit{or} implicitly encouraging of specific politics (nuclear power and authoritarianism, solar power and democracy, etc.).
    
%\begin{figure}[h!]
%\centering
%\includegraphics[scale=1.7]{universe}
%\caption{The Universe}
%\label{fig:universe}
%\end{figure}
%
%\section{Conclusion}
%``I always thought something was fundamentally wrong with the universe'' \citep{adams1995hitchhiker}
%
\bibliographystyle{plain}
\bibliography{refs}
\end{document}
